\section{Stato dell'arte}

\subsection{Ontologie e Sistemi di Tipi}
\begin{frame}[containsverbatim]{Ontologie e Sistemi di Tipi}
	Le ontologie usate nel Web Semantico possono essere comparate sotto certi aspetti ai sistemi di tipi. Ci basiamo sull'intuizione di \blue{Thierry Despeyroux} per trattare questa parte:
	
	\begin{columns}
		\begin{column}[T]{0.45\textwidth}
		\begin{center}
			\blue{Ontologia OWL}
		\end{center}
			Raggruppa nodi dei grafi RDF con le stesse proprietà in insiemi chiamati \blue{concetti};
			\begin{example}
%				I nodi che sono \texttt{Studenti} devono essere \textit{iscritti} a un \texttt{CdL}.
				$\texttt{Studente}~\sqsubseteq~\exists iscrittoA.\texttt{CdL}$
			\end{example}
%			\item Definisce relazioni tassonomiche fra concetti (\blue{sotto concetto})
%			\item permette di specificare asserzioni su individui tramite \textsc{A-Box} e attraverso un \blue{reasoner} si possono inferire quelle implicite.
		
	\end{column}
	\hspace{0.05\textwidth}
	\begin{column}[T]{0.45\textwidth}
		\begin{center}
			\blue{Sistema di tipi}
		\end{center}
		Raggruppa espressioni con le stesse proprietà in insiemi chiamati \blue{tipi};
		\begin{example}
			\begin{minted}{haskell}
data Nat = Succ Nat | Zero
			\end{minted}
		\end{example}
%		\begin{itemize}
%			\item Raggruppa espressioni con stesse proprietà in insiemi chiamati \blue{tipi};
%			\item Può permettere di definire relazioni tassonomiche fra tipi (\blue{subtyping})
%			\item permette di assegnare un tipo a un'espressione tramite \blue{regole di derivazione}.
%		\end{itemize}
	\end{column}
	\end{columns}
	
\end{frame}
\begin{frame}[containsverbatim]{Ontologie e Sistemi di Tipi}
	Le ontologie usate nel Web Semantico possono essere comparate sotto certi aspetti ai sistemi di tipi. Ci basiamo sull'intuizione di \blue{Thierry Despeyroux} per trattare questa parte:
	
	\begin{columns}
		\begin{column}[T]{0.45\textwidth}
			\begin{center}
				\blue{Ontologia OWL}
			\end{center}			
			Definisce relazioni tassonomiche fra concetti (\blue{subconcept})
			\begin{example}
				$\texttt{$\texttt{CdLUG }\sqsubseteq \texttt{CdL}$}$
				$\texttt{StudenteUG}~\sqsubseteq~\texttt{Studente}$\\
				\texttt{StudenteUG}~$\sqsubseteq$~\texttt{$\exists iscrittoA.\texttt{CdLUG}$}
			\end{example}
			
		\end{column}
		\hspace{0.05\textwidth}
		\begin{column}[T]{0.45\textwidth}
			\begin{center}
				\blue{Sistema di tipi}
			\end{center}
			Permette di definire relazioni tassonomiche fra tipi (\blue{subtyping})
			\begin{example}
				Nel sistema di tipi di Java, \mbox{\texttt{PosInteger extends Integer}} permette di definire la classe di interi \blue{positivi} come sottotipo di \texttt{Integer}.
			\end{example}
				%			\item permette di assegnare un tipo a un'espressione tramite \blue{regole di derivazione}.
				%		\end{itemize}
		\end{column}
	\end{columns}
	
\end{frame}

\begin{frame}[containsverbatim]{Ontologie e Sistemi di Tipi}
	Le ontologie usate nel Web Semantico possono essere comparate sotto certi aspetti ai sistemi di tipi. Ci basiamo sull'intuizione di \blue{Thierry Despeyroux} per trattare questa parte:
	\begin{columns}[T]
		\begin{column}[T]{0.45\textwidth}
			\begin{center}
				\blue{Ontologia OWL}
			\end{center}			
			Permette di specificare asserzioni su individui tramite \textsc{A-Box} e attraverso un \blue{reasoner} si possono inferire quelle implicite.
			\begin{example}
				$\text{Informatica} : \texttt{CdL}$.\\
				$(\text{Alessio}, \text{Informatica}) : \textit{iscrittoA}$.\\
				\blue{$\implies \text{Alessio} : \texttt{Studente}$}.
			\end{example}
			
		\end{column}
		\hspace{0.05\textwidth}
		\begin{column}[T]{0.45\textwidth}
			\begin{center}
				\blue{Sistema di tipi}
			\end{center}
			Permette di assegnare un tipo a un'espressione tramite \blue{regole di derivazione}.
			\begin{example}
				\begin{center}
					\myruleN{\Gamma \vdash n : Nat}{\Gamma \vdash Succ\ n : Nat}{T-SUCC}
				\end{center}
			\end{example}
	\end{column}
\end{columns}

\end{frame}

\begin{frame}{Ontologie e Sistemi di Tipi}
Da questo confronto si potrebbe pensare a modi per usare i sistemi di tipi per controllare certe proprietà delle ontologie. Al meglio delle nostre conoscenze, in letteratura ci sono stati pochi contributi di questo genere:
\begin{itemize}
	\item R.~Dapoigny e P.~Barlatier, \blue{teoria di tipi intuizionista} (\textsc{K-DTT}) per rappresentare ontologie, utilizzando i tipi dipendenti per la loro capacità di supportare un'elevata espressività e un ragionamento potente. La teoria presentata è più espressiva e complessa della normale logica dei predicati.
	\item M.~Leinberger, \blue{$\lambda_{DL}$} per controllare a tempo di compilazione se una query SPARQL può avere un risultato o meno, utilizzando i tipi (spiegato più avanti). Questo permette di evitare errori inattesi a tempo di esecuzione.
\end{itemize}
\end{frame}

\subsection{Ontologie e Linguaggi Funzionali}
\begin{frame}{Linguaggi Funzionali nel Web Semantico\\{\large Perchè non vengono usati?}}
	La maggior parte degli strumenti e reasoner presenti in letteratura viene sviluppato usando linguaggi di programmazione \blue{object-oriented}, in particolare Java.
	\begin{itemize}
		\item I concetti di un'ontologia sono facilmente rappresentabili come classi e gli individui come oggetti.
		\item La relazione di subtyping è molto simile a quella di subconcept.
		\item Da più di dieci anni è stata sviluppata OWL API, che facilita il compito di scrivere strumenti per le ontologie in Java.\\
		$\implies$ molti reasoner sono sviluppati in Java (es. \blue{HermiT})		
	\end{itemize}
	
	Tutta l'inferenza e i controlli su proprietà dell'ontologia in questi approcci vengono fatti a tempo di esecuzione.
\end{frame}

\begin{frame}{Linguaggi Funzionali nel Web Semantico}
	Siamo convinti che i linguaggi funzionali con tipi statici permettano di manipolare un'ontologia più facilmente, in maniera modulare e sicura.
	
	P. Hitzler, \blue{(f OWL)}: un editor e API per le ontologie OWL (2).
	\begin{itemize}
		\item Sviluppato in \blue{Clojure}. Altamente portabile, usa la JVM per compilare;
		\item usa la \blue{lazy evaluation}, tipica dei linguaggi funzionali, e una \blue{struttura omogenea} per velocizzare e semplificare la creazione e modifica di un'ontologia;\\
		\blue{$\implies$ le funzioni scritte per singoli assiomi o espressioni spesso funzionano in modo identico su qualsiasi parte di un'ontologia, persino sull'intera ontologia stessa.}
		\item anche se ha prestazioni peggiori nella lettura di un'ontologia da file, eccelle nella trasformazione in forma normale negata degli assiomi dell'ontologia. % Portare un'ontologia in questa forma permette di confrontarne l'espressività con altre ontologie, permettendo di valutarne la bontà per il suo riutilizzo.
	\end{itemize}
	
\end{frame}
