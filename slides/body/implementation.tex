\section{Implementazione $\lambda_{DL}$}
\subsection{Query SPARQL CQ}
\begin{frame}{Query SPARQL CQ}
    \begin{block}{Definizione}
        Una query SPARQL CQ è un sistema di interrogazione per i grafi RDF, in cui le clausole sono in forma congiuntiva
    \end{block}
    \begin{block}{Query abitabile}
        In una query SPARQL, il concetto di abitabilità identifica
le query che possono ritornare almeno un elemento
    \end{block}
    
    \begin{example}
        Query abitabile: \\
        query x $\leftarrow$ x type Pizza
    \end{example}
    \begin{example}
        Query non abitabile: \\
        query x $\leftarrow$ x type Pizza $\land$ x type Gelato
    \end{example}
 
\end{frame}

\subsection{$\lambda_{DL}$}
\begin{frame}[containsverbatim]{$\lambda_{DL}$}
\blue{Martin Leinberger} nella sua tesi di dottorato propone il \blue{$\lambda_{DL}$}, un \blue{$\lambda$-calcolo tipato}
per poter lavorare con ontologie e avere controlli statici sulle interazioni con esse. Le novit\`a principali sono:
\begin{itemize}
\item Liste e operazioni su di esse come head e tail
\begin{example}
\begin{minted}{haskell}
["L", "a", "y", "t", "o"]
\end{minted}
\end{example}
\item Record e la proiezione su una label
\begin{example}
\begin{minted}{haskell}
{y = 10, i = 0}
\end{minted}
\end{example}
\item Query SPARQL CQ
\begin{example}
\begin{minted}{haskell}
query x <- x type Student
\end{minted}
\end{example}
\end{itemize}
\end{frame}

\begin{frame}[containsverbatim]{$\lambda_{DL}$}
    Per dare un \blue{tipo} ai nodi del grafo RDF, ottenuti dall'esecuzione delle Query SPARQL CQ, \`e stato aggiunta ai tipi standard la \blue{Concept Expression}
    della logica descrittiva:
    ~\\
    \begin{example}
        $K_4 = $\{Student $\sqsubseteq$ Person
        \\Professor $\sqsubseteq$ Person\}
        \begin{minted}[escapeinside=||,mathescape=true, autogobble]{ocaml}
            (head (query x <- x type Student)).x : Student 
        \end{minted}
    \end{example}
    ~\\
Siccome \blue{Student~$\sqsubseteq$~Person} l'espressione ha anche tipo \blue{Person}.
\end{frame}

\subsection{Sull'implementazione del $\lambda_{DL}$}
\begin{frame}{Sull'implementazione del $\lambda_{DL}$}
    Questa implementazione del calcolo di Leinberger prende ispirazione dell'implementazione in Scala dell'autore, ma è originale ed è la prima implementazione di $\lambda_{DL}$ che viene fatta in OCaml. 
    \\\`{E} strutturata in due moduli:
    \begin{itemize}
        \item Modulo OCaml: scritto nell'omonimo linguaggio funzionale. 
        \item Modulo Reasoner: scritto in Java.
    \end{itemize}
\end{frame}

\subsection{Sull'implementazione del $\lambda_{DL}$: OCaml Module}
\begin{frame}{Sull'implementazione del $\lambda_{DL}$: OCaml Module}
    Il modulo OCaml si occupa di parsificare la query, inferire gli assiomi e invocare il modulo Reasoner.
    \begin{block}{Tipi di supporto}
        Il modulo si avvale di alcuni Tipi per rappresentare l'informazione in OCaml:
        \begin{itemize}
            \item Il tipo Query, viene usato per rappresentare nel programma le query SPARQL.
            \item Il tipo ClassExpression per rappresentare una concept expression.
            \item Il tipo Axiom, rappresenta un assioma di Leinberger. 
        \end{itemize}
    \end{block}
    \begin{block}{Manchester OWL Syntax}
    Inoltre viene utilizzata come lingua veicolare di dialogo fra il modulo OCaml e il modulo Reasoner.
    \end{block}
\end{frame}

\begin{frame}{Sull'implementazione del $\lambda_{DL}$: OCaml Module}
    % 1 query e tipo query esempio
    % 2 ocaml trasforma query in tipo query 

    \begin{itemize}
        \item Parsifica una query in formato stringa in un tipo Query
        \begin{example}
            query x $\leftarrow$ x type Pizza $\Rightarrow$
            Q (x, SP(V x, TYPE, A Pizza))
        \end{example}
        \item Inferisce gli assiomi di Leinberger sul tipo Query
        \begin{example}
            Q (x, SP(V x, TYPE, A Pizza))
            $\Rightarrow$ 
            (x, Atomic(Pizza))
        \end{example}
        \item Traduce gli assiomi in sintassi di Manchester e li inoltra al Reasoner
        \begin{example}
            (x, Atomic(Pizza))
            $\Rightarrow$ 
            x : Pizza
        \end{example}
    \end{itemize}
\end{frame}

\subsection{Sull'implementazione del $\lambda_{DL}$: Reasoner Module}
\begin{frame}{Sull'implementazione del $\lambda_{DL}$: Soddisfacibilità di un ontologia}
    \begin{block}{Soddisfacibilità di un ontologia}
        Data un ontologia il reasoner HermiT decide se l'ontologia è soddisfacibile, cioè decide se esiste un modello che soddisfa l'ontologia.
     \end{block}
    \begin{example}
        Per esempio, l'ontologia che presenta un unico assioma di sottoclasse tra il concetto atomico X e la congiunzione tra Pizza e Gelato non è soddisfacibile.
    \end{example}
\end{frame}

\begin{frame}{Sull'implementazione del $\lambda_{DL}$: Reasoner Module}
    Il modulo Reasoner si occupa di verificare la soddisfacibilità degli assiomi di Leinberger, avvalendosi del un reasoner HermiT.
    \begin{block}{Comportamento del modulo}
        \begin{itemize}
            \item Il modulo aggiunge gli assiomi di Leinberger ricevuti dal modulo OCaml all'ontologia.
            \item HermiT decide se l'ontolgia è soddisfacibile.
            \item Se lo è vuol dire che la query è abitata, altrimenti no. 
        \end{itemize} 
    \end{block}
\end{frame}
\subsection{Sistema di Tipi del $\lambda_{DL}$}
\begin{frame}[containsverbatim]{Sistema di Tipi del $\lambda_{DL}$}
    Per implementare il sistema di tipi\footnote{\scriptsize Abbiamo preso spunto da "Type and Programming Languages" di Benjamin C. Pierce \\} abbiamo definito i \blue{termini} del linguaggio:
    \begin{block}{data type per i Termini del $\lambda_{DL}$}
    \begin{minted}[escapeinside=&&,mathescape=true, autogobble]{ocaml}
    type Term =
        | TmRecord of info * (string * term) list 
        | TmProj of info * term * string 
        | TmNil of info * ty 
        | TmCons of info * term * term 
        | TmIsNil of info * term 
        | TmHead of info * term 
        | TmTail of info * term  
        | TmQuery of info * var * gp
        | TmRoleProj of info * term * role
        | TmNode of info * string
    \end{minted}
    \end{block}
\end{frame}

\begin{frame}[containsverbatim]{Sistema di Tipi del $\lambda_{DL}$}
Bisogna definire i \blue{tipi} del linguaggio
    \begin{block}{data type per i tipi del $\lambda_{DL}$}
    \begin{minted}[escapeinside=&&,mathescape=true, autogobble]{ocaml}
    type Ty =
          TyTop 
        | TyBool 
        | TyRecord of (string * ty) list 
        | TyArr of ty * ty 
        | TyNat 
        | TyList of ty
        | TyCe of ce
    \end{minted}
    \end{block}
    e definire la funzione che preso un termine restituisce il suo tipo:
    \begin{block}{Funzione typeof}
    \begin{minted}[escapeinside=&&,mathescape=true, autogobble]{ocaml}
    typeof :: Context -> Term -> Ty
    \end{minted}
    \end{block}
\end{frame}

\begin{frame}[containsverbatim]{Sistema di Tipi del $\lambda_{DL}$}
la funzione \code{typeof} \`e implementata seguendo le regole di tipo. Un esempio interessante \`e la regola per dare un tipo alle query:
~\\
\begin{block}{implementazione [T-QUERY]}
\begin{minted}[escapeinside=||,mathescape=true, autogobble]{ocaml}
    let rec typeof ctx t =
        match t with
        |$\vdots $|
        TmQuery(fi, var, gp) ->
            if allSatisfiable(axioms(gp), var) then
            TyList(TyRecord(var, TyCe(Atomic var))) else
                error fi "axioms unsatisibale"
        |$\vdots$|
\end{minted}
\end{block}
\end{frame}

\begin{frame}[containsverbatim]{Sistema di Tipi del $\lambda_{DL}$}
Per il subtyping abbiamo definito la funzione \code{subtype}:
\begin{block}{Funzione subtype}
\begin{minted}{ocaml}
subtype ::  Context -> Ty -> Ty -> Bool
\end{minted}
\end{block}
Ad esempio per il subtyping tra Concept Expression abbiamo:  
\begin{block}{implementazione [S-CONCEPT]}
\begin{minted}[escapeinside=&&,mathescape=true, autogobble]{ocaml}
    let rec subtype ctx tyS tyT =
        tyeqv ctx tyS tyT ||
        match (tyS,tyT) with
        &$\vdots $&
        | (TyCe(ceS1),TyCe(ceT1)) -> subconcept ceS1 ceT1
        &$\vdots$&
\end{minted}
\end{block}
\end{frame}

\begin{frame}{Sistema di Tipi del $\lambda_{DL}$}
    Perch\`e implementare $\lambda_{DL}$?
    \begin{itemize}
        \item Ci ha aiutato a capire meglio come i sistemi di tipi statici possono essere utilizzati con le ontologie.\\~\\
        \item Abbiamo concluso che OCaml \`e un linguaggio adatto per l'implementazione di linguaggi di programmazione e per interfacciarsi con altri linguaggi per fare reasoning su ontologie. \\~\\
        \item \`E stato un ottimo esercizio di programmazione funzionale.\\~\\
        \item Ci ha spinto ad approfondire le ontologie, la logica descrittiva e i linguaggi di programmazione.\\~\\
        \item I risultati ottenuti incoraggiano lo studio su possibili direzioni future sull'utilizzo di linguaggi funzionali e sistemi tipi statici per il Web Semantico.
    \end{itemize}
\end{frame}