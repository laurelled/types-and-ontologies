\section{Programmazione funzionale}
    La programmazione funzionale è un paradigma che permette di scrivere in modo naturale programmi di semplice lettura e dubugging grazie sopratutto all'assenza di side-effect nelle funzioni.
    I side-effect o in italiano effetti collaterali, si verificano quando si modifica lo stato di variabili al di fuori dello scope locale. Ad esempio una semplice funzione in Java\footnote{\url{https://www.java.com/}}
    \begin{minted}{java}
    void move(Poitn p){
        p.x++;
        p.y++;
    }
    \end{minted} 
    che prende in input un oggetto \code{Point p} con attributi \code{float x} e \code{float y}. Quando eseguito, essendo \code{p} un oggetto, viene passato per riferimento
    quindi provocando la modifica dello stato dell'oggetto anche al di fuori dello scope di \code{void move(Point p)}. Linguaggi funzionali come Haskell\footnote{https://www.haskell.org/} non
    permetto la creazione di funzioni inpure come \code{move}, ma costringono il programmatore a non produrre side-effect.
    \begin{minted}{haskell}
    type Point = (Float, Float)
    move :: Point -> Point
    move (x, y) = (x + 1, y + 1)
    \end{minted}
    In questo caso \code{move} ritorna un nuovo \code{Point} con gli attributi \code{x} e \code{y} incrementati di \code{1}, senza modificare lo stato di alcuna
    variabile.
    \\I linguaggi funzionali sono molto versatili sopratutto grazie alla possibilità di avere funzioni di ordine superiore, ovvero delle funzioni che come parametro
    hanno altre funzioni. Un caso tipico di utilizzo di funzioni di ordine superiore è l'implementazione della funzione \code{map}.
    \begin{minted}{haskell}
    map :: (a -> b) -> [a] -> [b]
    map _ []     = []
    map f (x:xs) = f x : map f xs
    \end{minted}
    \code{map} prende in input una funzione \code{(a -> b)} e una lista di oggetti di tipo \code{a} (\code{[a]}) e ritorna una lista di oggetti di tipo \code{b} (\code{[b]}). 
    la funzione passata come parametro, \code{f}, viene applicata ricorsivamente ad ogni elemento della lista \code{f x : map f xs} generando così la lista risultante. 
    Nei due esempi precedenti abbiamo usato silenziosamente un'altra feature molto utilizzata nei linguaggi funzionali: il pattern matching.
    \code{move (x, y) = (x + 1, y + 1)} confronta l'argomento della funzione con i pattern. In questo caso l'argomento è di tipo \code{Point} ovvero
    una coppia di \code{Float}, quindi quando si scrive \code{move (x, y) = ...}, \code{x} e \code{y} avranno i valori dei campi dell'oggetto \code{Point}
    che potremo liberamente usare nel body della funzione (\code{(x + 1, y + 1)}).
    La funzione \code{map} sfrutta il pattern mathing per distinguere il caso in cui la lista è vuota \code{map _ []} (caso base della ricorsione), dal caso in cui
    bisogna applicarre la ricorsione sulla lista \code{map f (x:xs)}. 
    \\I linguaggi funzionali sono spesso accompagnati da uno stretto ma flessibile sistema di tipi
    che permette di catturare a tempo di compilazione molti errori che altrimenti ritroveremmo a runtime. I data type creati in linguaggi come Haskell o OCaml\footnote{https://ocaml.org/}
    permettono di creare tipi al programmatore dandogli così molta flessibilità mantenendo i controlli statici. per esempio con il tipo \code{Nat}
    \begin{minted}{haskell}
        data Nat = zero | Succ Nat
    \end{minted}
    si sono definiti i numeri naturali con cui, attraverso il pattern matching, è possibile creare semplicemente funzioni che operano su di essi:
    \begin{minted}{haskell}
        (+) :: Nat -> Nat -> Nat
        zero     + y = x
        (Succ x) + y = succ (x + y) 
    \end{minted}
    la funzione \code{(+)} che esegue semplicemente l'addizione su due numeri naturali, anche se molto in piccolo, mostra l'eleganza, la chiarezza e le potenzialita dei linguaggi funzionali.