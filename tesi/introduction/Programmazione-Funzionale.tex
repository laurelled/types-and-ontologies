\section{Programmazione funzionale}
Il paradigma funzionale è un paradigma linguistico permette di scrivere in modo naturale programmi di facile comprensione e debugging grazie sopratutto all'assenza di \emph{side-effect} nelle funzioni.
I \emph{side-effect} o in italiano \emph{effetti collaterali}, si verificano quando si modifica lo stato di variabili al di fuori dello scope locale. Ad esempio consideriamo una semplice funzione Java:
\begin{minted}{java}
	void move(Point p){
		p.x++;
		p.y++;
	}
\end{minted} 
che prende in input un oggetto \code{Point p} con attributi \code{float x} e \code{float y}. Quando eseguito, essendo \code{p} un oggetto, viene passato per riferimento
quindi provocando la modifica dello stato dell'oggetto anche al di fuori dello scope di \code{void move(Point p)}. Linguaggi funzionali come Haskell\footnote{https://www.haskell.org/} non
permettono la creazione di funzioni impure come \code{move}, ma costringono il programmatore a non produrre side-effect.
\begin{minted}{haskell}
	type Point = (Float, Float)
	move :: Point -> Point
	move (x, y) = (x + 1, y + 1)
\end{minted}
In questo caso \code{move} ritorna un nuovo \code{Point} con gli attributi \code{x} e \code{y} incrementati di \code{1}, senza modificare lo stato di alcuna
variabile.
\\I linguaggi funzionali sono molto versatili sopratutto grazie alla possibilità di avere funzioni di ordine superiore, ovvero delle funzioni che come parametro
hanno altre funzioni. Un caso tipico di utilizzo di funzioni di ordine superiore è l'implementazione della funzione \code{map}.
\begin{minted}{haskell}
	map :: (a -> b) -> [a] -> [b]
	map _ []     = []
	map f (x:xs) = f x : map f xs
\end{minted}
\code{map} prende in input una funzione \code{(a -> b)} e una lista di oggetti di tipo \code{a} (\code{[a]}) e ritorna una lista di oggetti di tipo \code{b} (\code{[b]}). 
la funzione passata come parametro, \code{f}, viene applicata ricorsivamente ad ogni elemento della lista \code{f x : map f xs} generando così la lista risultante. 
Nei due esempi precedenti abbiamo usato silenziosamente un'altra feature molto utilizzata nei linguaggi funzionali: il pattern matching.
\code{move (x, y) = (x + 1, y + 1)} confronta l'argomento della funzione con i pattern. In questo caso l'argomento è di tipo \code{Point} ovvero
una coppia di \code{Float}, quindi quando si scrive \code{move (x, y) = ...}, \code{x} e \code{y} avranno i valori dei campi dell'oggetto \code{Point}
che potremo liberamente usare nel body della funzione (\code{(x + 1, y + 1)}).
La funzione \code{map} sfrutta il pattern matching per distinguere il caso in cui la lista è vuota \code{map _ []} (caso base della ricorsione), dal caso in cui
bisogna applicare la ricorsione sulla lista \code{map f (x:xs)}. 
\\I linguaggi funzionali sono spesso accompagnati da uno stretto ma flessibile sistema di tipi statico, che permette di catturare a tempo di compilazione molti errori che altrimenti ritroveremmo a runtime. I data type creati in linguaggi come Haskell o OCaml\footnote{https://ocaml.org/}
permettono di creare tipi al programmatore dandogli così molta flessibilità, mantenendo i controlli statici. per esempio con il tipo \code{Nat}
\begin{minted}{haskell}
	data Nat = zero | Succ Nat
\end{minted}
si sono definiti i numeri naturali con cui, attraverso il pattern matching, è possibile creare semplicemente funzioni che operano su di essi:
\begin{minted}{haskell}
	(+) :: Nat -> Nat -> Nat
	zero     + y = x
	(Succ x) + y = succ (x + y) 
\end{minted}
la funzione \code{(+)} che esegue semplicemente l'addizione su due numeri naturali, anche se molto in piccolo, mostra l'eleganza, la chiarezza e le pontenzialita dei linguaggi funzionali.

Per i motivi descritti sopra, i linguaggi funzionali sono più vicini alla rappresentazione e manipolazione simbolica dei dati e quindi sembrano adatti per la costruzione di strumenti per supportare il reasoning nell'ambito della knowledge representation \ref{chap:Implementazione}.
Tuttavia, l'innovazione più interessante potrebbe nascere dalla progettazione di nuovi linguaggi con sistemi di tipi statici, rappresentanti nuovi modelli 
di reasoning. Grazie alla descrizione tramite le DL, le ontologie permettono inferenze automatiche di proprietà sulla tassonomia definita e sui dati che ne 
fanno uso. Per ragioni di efficienza, però, nel campo del Web Semantico la tendenza attuale è quella di utilizzare meno le inferenze formali sulle ontologie, 
che permettono ragionamenti sofisticati ma poco efficienti dal punto di vista della complessità temporale, in favore dell'utilizzo delle sole asserzioni 
della \textsc{\itshape A-Box} (di solito rappresentate come grafi RDF) che diventano dati di input per sistemi sub-logici, come quelli basati sul machine 
learning, guadagnando in efficienza, ma perdendo in espressività (si noti infatti che l'uso della sola A-Box non è ragionamento ontologico in senso stretto). 
Questo è dato dal fatto che il reasoning sulle ontologie, per quanto formalmente decidibile, ha una complessità elevata e questo rende il ragionamento logico 
inutilizzabile per via della mole di dati che le applicazioni hanno sempre più necessità di usare \cite{baader2017introductionDL}. Tuttavia, la nostra 
intuizione è che una direzione per ritornare a svolgere un ragionamento formale e riproducibile, recuperando parte dell'espressività persa senza perdere 
troppo in efficienza, sia quella di studiare calcoli e linguaggi funzionali di rappresentazione delle ontologie con sistemi di tipi statici per fare 
inferenza di proprietà. Uno dei vantaggi dei sistemi di tipi statici è che le inferenze si possono appunto fare staticamente, quindi prima del running di un 
sistema, evitando di appesantire il runtime con le inferenze. Si tratta di stabilire quali proprietà possono essere delegate a sistemi di tipi di questo 
genere e se, come in effetti pensiamo, questi possano avere effetti benefici con grandi moli di dati. In questo lavoro, abbiamo analizzato in questo senso il 
sistema ... presentato nella tesi di dottorato di Martin Leinberger \cite{leinbergerphdthesis}, che trattiamo nel Capitolo \ref{chap:Implementazione}. Inoltre, il Capitolo %\ref{chap:Future Work} 
presenta alcune direzioni di ricerca che abbiamo intravisto.
