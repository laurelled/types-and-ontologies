\newpage
\section{Scopo della tesi}
Questo lavoro è uno studio preliminare su possibili proposte nella direzione di \textsl{fornire strumenti formali basati su linguaggi funzionali staticamente tipati} per il Web Semantico. Utilizzare strumenti logici come i sistemi di tipi per fare ragionamento potrebbe sembrare una direzione di ricerca che va nel senso contrario rispetto alla preferenza corrente, ossia dell'applicazione di tecniche sub-logiche molto più efficienti legate soprattutto al machine learning. Sebbene l'efficienza sia un aspetto chiave di un processo di computazione, non è l'unico fattore.
\noindent
Questo lavoro offre:
\begin{enumerate}[I.]

	\item una panoramica significativa, ancorché incompleta, dello stato d'arte relativo alla ricerca sull'uso dei linguaggi funzionali e dei sistemi di tipi statici nel contesto del Web Semantico e della rappresentazione di conoscenza. Inoltre, illustriamo alcuni argomenti sui quali, al meglio della nostra conoscenza, non esistono ancora modelli e strumenti che fanno uso del paradigma funzionale e dei tipi, ma che potrebbero trarre vantaggio da questo utilizzo. Di questa parte mi sono occupato principalmente io.
	\item  un esperimento di studio e implementazione di una proposta di linguaggio funzionale con tipi statici nell'ambito del Web Semantico, ovvero il sistema $\lambda_{DL}$ di Martin Leinberger introdotto nella sua tesi di dottorato "Type-safe Programming for the Semantic Web" \cite{leinbergerphdthesis}. Principalmente, Emanuele Rovaretto e Andrea Zito hanno portato avanti questo esperimento.
	\item proposte di possibili direzioni future per l'utilizzo di linguaggi funzionali staticamente tipati per programmare applicazioni che manipolano le ontologie e lo studio di sistemi di tipi statici che garantiscano proprietà interessanti ai programmi che svolgono computazioni basate su ontologie, certificati dal sistema di tipi stesso. Questa parte è scaturita da un lavoro corale, supportato da alcuni proficui scambi con Marco Antonio Stranisci, Rossana Damiano e Antonio Lieto del Dipartimento di Informatica dell'Università di Torino.
\end{enumerate}
\noindent
Per quanto riguarda le direzioni future si noti che, attualmente, gran parte dei processi di ragionamento basati su un'ontologia sono eseguiti a run-time, cioè durante l'esecuzione del programma, usando librerie Java come OWL API \cite{OWLAPI} per rappresentare gli assiomi ontologici nei linguaggi di programmazione object-oriented e sfruttarli per fare inferenza. Se incorrono errori a run-time dovuti all'ontologia, non c'è alcuna garanzia formale che il programma possa terminare dando il risultato desiderato. Proporre, quindi, di sfruttare i sistemi di tipi statici in cui l'utilizzo dei tipi, controllati a tempo di compilazione, potrebbe essere una risposta ai problemi di efficienza per certe proprietà che avrebbe senso controllare a priori, e/o nel caso di grandi quantità di dati. La tesi di dottorato di Martin Leinberger \cite{leinbergerphdthesis} che abbiamo approfondito in questo lavoro, focus principale dell'implementazione nel \autoref{chap:Implementazione}, va in questa direzione, poiché propone un lambda-calcolo con tipi statici per decidere a tempo di compilazione se una query SPARQL è abitata, ovvero se produrrà un risultato quando interpretata, per assicurare il suo utilizzo a tempo di esecuzione. Nel \autoref{chap:FutureWork} si trovano gli approfondimenti su alcune direzioni di ricerca future.

