\section{Semantic Web}
Il Web Semantico fornisce un framework comune che consente la condivisione e il riutilizzo dei dati al di là dei confini delle applicazioni.
 È uno sforzo collaborativo guidato dall'organizzazione W3C\footnote{\url{https://www.w3.org/about/}} con la partecipazione di un gran numero di ricercatori 
 e partner industriali. Gli obiettivi principali del Web Semantico \cite{berners2001semantic, hitzler2021review} sono due:
\begin{enumerate}[I)]
	\item creare una rete di dati interconnessi, in contrapposizione all'attuale Web basato sui documenti;
	\item permettere che una macchina possa comprendere le informazioni disponibili sul Web senza intervento umano.
\end{enumerate}
Per soddisfare entrambi, si è resa necessaria l'introduzione di annotazioni espressive che spieghino la correlazione fra i dati, ed è per questo che W3C ha 
introdotto il \textit{Resource Description Framework} (RDF) \cite{RDFspecification}. RDF è uno standard che permette la codifica, lo scambio e il riutilizzo 
di metadati (dati che descrivono altri dati), strutturandoli come dichiarazioni di triple soggetto, predicato e oggetto. In questo modo RDF permette di 
rappresentare dei grafi, i cui nodi e vertici rappresentano informazioni presenti nel web (chiamate \textit{risorse}) e sono identificate dagli IRI, 
identificatori unici di risorse che svolgono la stessa funzione degli URL per i documenti web. Essendo unici, essi consentono di associare ai nodi di due 
grafi diversi la stessa risorsa. Per condividere la terminologia (e.g. cosa s'intende per "Studente") è stato proposto da W3C un ulteriore standard, le 
ontologie OWL. In generale, le ontologie sono rappresentazioni formali, condivise ed esplicite di una concettualizzazione di un dominio di interesse 
\cite{goy2015ontologies} in maniera complessa e strutturata. OWL (\textit{Ontology Web Language}) è un linguaggio altamente espressivo e formale basato su 
logiche appartenenti al campo della knowledge representation (KR), le logiche descrittive (DL) \cite{baader2017introductionDL}. La forma logica delle 
ontologie OWL di definire un ragionamento automatico basato su inferenze, cioè dedurre nuove informazioni basandosi su quelle che sappiamo essere vere.
La definizione del dominio d'interesse tramite ontologie permette di aggiungere ulteriore struttura ai dati espressi nei grafi RDF, specificandone uno schema. 
In questo senso, a volte, ci si riferisce alle ontologie come a dei “sistemi di tipi” per tali dati. Un sistema di tipo, nei linguaggi di programmazione, 
è un sistema logico che permette di assegnare a ogni termine un tipo, che identifica le caratteristiche e i valori possibili di quel termine. Storicamente, 
le ontologie basate su DL comprendono almeno due tipologie di asserzioni:
\begin{enumerate}[i)]
	\item dichiarazioni di concetti, che vanno a svolgere il ruolo di "tipo" per i nodi RDF, e la loro gerarchia. L'insieme di tutte queste definizioni 
    è detto \textsc{\itshape T-Box}. Essa contiene quindi tutta la parte di terminologia, ovvero le \textit{condizioni necessarie e sufficienti} per un 
    elemento di far parte di un concetto.
	\item asserzioni sugli individui o di sussistenza di una proprietà. Ad esempio, dire che \textsl{Elena} è una Persona, oppure che la 
    \textsl{ Pizza margherita} ha come ingrediente \textsl{Pomodoro}. L'insieme di queste asserzioni è detto \textsc{\itshape A-Box}, e rappresenta 
    tutte le "istanze" rilevanti per la realtà descritta.
\end{enumerate}
Grazie alla descrizione tramite formalismi logici, le ontologie permettono inferenze automatiche sulla tassonomia definita e sui dati che ne fanno uso. 
Per ragioni di efficienza, però, nel campo del Web Semantico è diventato d'uso comune abbandonare il ragionamento formale delle ontologie per passare ai 
più efficienti ragionamenti sub-logici dei modelli di machine learning, usando come input le asserzioni presenti nell'\textsc{\itshape A-Box} rappresentate 
come grafi RDF, meno espressivi e quindi efficienti. Questo è dato dal fatto che la rappresentazione logica, per quanto formalmente decidibile, ha una 
complessità talmente elevata da rendere il ragionamento logico inutilizzabile per la mole di dati che le applicazioni hanno necessità di usare 
\cite{baader2017introductionDL}. Tuttavia, la nostra convinzione è che possano esserci delle potenzialità da sfruttare nei sistemi di tipi statici nel 
campo della knowldege representation, in modo da ritornare a svolgere un ragionamento formale e riproducibile, recuperando parte dell'efficienza e 
dell'espressività persa.