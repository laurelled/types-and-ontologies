Il World Wide Web è diventato una ricca rete di informazioni, ma per lungo tempo ha sofferto del difetto di essere pensato per la sola fruizione umana. Il linguaggio principale di mark-up per la rappresentazione delle centinaia di pagine che visitiamo ogni giorno, HTML, descrive, assieme ad altri formalismi, soltanto la struttura di impaginazione e visualizzazione grafica (layout) del contenuto. Senza, quindi, una struttura ulteriore che rappresenti la semantica dei dati presentati, un agente artificiale non può comprendere la maggior parte delle informazioni presenti in rete. Il Web Semantico è la risposta a tale esigenza, poiché fornisce un framework comune che consente la condivisione e il riutilizzo dei dati al di là dei confini delle applicazioni. È uno sforzo collaborativo guidato dall'organizzazione W3C\footnote{\url{https://www.w3.org/about/}} con la partecipazione di un gran numero di ricercatori e partner industriali. Gli obiettivi principali del Web Semantico \cite{berners2001semantic, hitzler2021review} sono due:
\begin{enumerate}[I)]
	\item creare una rete di dati interconnessi, in contrapposizione all'attuale Web basato sui documenti;
	\item permettere che una macchina possa comprendere le informazioni disponibili sul Web senza intervento umano.
\end{enumerate}
Per soddisfare entrambi, si è resa necessaria l'introduzione di annotazioni espressive che spieghino la correlazione fra i dati, ed è per questo che W3C ha introdotto il formato \textit{Resource Description Framework} (RDF) \cite{RDFspecification}. RDF è uno standard che permette la codifica, lo scambio e il riutilizzo di metadati (ossia, dati che descrivono altri dati), strutturandoli come dichiarazioni di triple "soggetto, predicato, oggetto". In questo modo RDF permette di rappresentare dei grafi, i cui nodi e archi rappresentano rispettivamente le informazioni presenti nel Web e i loro collegamenti (chiamati \textit{risorse}) e sono identificate dagli IRI, identificatori unici di risorse che svolgono la stessa funzione degli URL per i documenti web. Essendo unici, essi consentono di associare ai nodi di due grafi diversi la stessa risorsa. Per condividere la descrizione strutturata di realtà di interesse (per esempio, cosa s'intende per "Università" e le sue componenti) è stato proposto da W3C un ulteriore standard, le ontologie OWL. In generale, le ontologie sono rappresentazioni formali, condivise ed esplicite di una concettualizzazione di un dominio di interesse \cite{goy2015ontologies} in maniera complessa e strutturata. OWL (\textit{Ontology Web Language}) è un linguaggio altamente espressivo e formale basato su logiche descrittive (DL), particolari linguaggi formali appartenenti  al campo della knowledge representation (KR). La forma logica delle ontologie OWL permette di definire un ragionamento automatico basato su inferenze, cioè dedurre nuove informazioni basandosi su quelle che sappiamo essere vere.

La definizione del dominio d'interesse tramite ontologie permette di aggiungere ulteriore struttura ai dati espressi nei grafi RDF, specificandone uno schema.
In questo senso, a volte, ci si riferisce alle ontologie come a dei “sistemi di tipi” per tali dati, ovvero a delle astrazioni che descrivono certe proprietà interessanti dei dati, per esempio il fatto che un certo nodo del grafo appartenga a un certo concetto (per un'introduzione sui sistemi di tipi nei linguaggi di programmazione, ai quali ci ispiriamo, si veda il libro~\cite{TypesAndProgrammingLanguages}). 

Storicamente, le ontologie basate su DL comprendono almeno due tipologie di asserzioni:
\begin{enumerate}[i)]
	\item dichiarazioni di concetti, che vanno a svolgere il ruolo di \textit{tipo} per i nodi RDF, e la loro gerarchia. L'insieme di tutte queste definizioni è detto \textsc{\itshape T-Box}. Essa contiene quindi tutta la parte di terminologia, ovvero le \textit{condizioni necessarie e sufficienti} per un elemento di far parte di un concetto.
	\item asserzioni sugli individui o di sussistenza di una proprietà. Ad esempio, dire che \textsl{Elena} è una \texttt{Persona}, oppure che la \textsl{Pizza Margherita} ha come ingrediente \textsl{Pomodoro}. L'insieme di queste asserzioni è detto \textsc{\itshape A-Box}, e rappresenta tutte le \textit{istanze} rilevanti per la realtà descritta.
\end{enumerate}