\chapter[Concetti preliminari]{Concetti preliminari}
Per poter considerare l'utilizzo dei tipi per il Web Semantico, è necessario analizzare lo stato dell'arte in cui si ritrova questo ambito di ricerca.
Questo capitolo si presta utile per apprendere il vocabolario utilizzato durante tutta la tesi, nonché fornire approfondimenti su nozioni che potrebbero essere date per scontate.
Verranno riassunte le basi dei linguaggi funzionali, così come le componenti del Web Semantico.
\section[Linguaggi funzionali]{Linguaggi funzionali}
\section[Web Semantico]{Web Semantico}
Il World Wide Web è ormai una gigantesca e consolidata rete di conoscenza, che però nasconde un difetto fondamentale per l'utilizzo di queste informazioni da parte di un agente artificiale. Infatti il linguaggio di rappresentazione, HTML, è stato pensato per la fruizione umana dei suoi contenuti piuttosto che di una macchina. Ciò rende difficile per quest'ultime capire il significato delle informazioni presenti nel web. Uno degli obiettivi del Web Semantico, presentato per la prima volta nel 2001 in \cite{berners2001semantic}, è cambiare questo paradigma human-centered, permettendo agli agenti artificiali di interpretare e processare la conoscenza presente senza alcun tipo di aiuto umano. È necessario descrivere le informazioni attraverso metadati espressivi, strutturandoli arbitrariamente, che ne spieghino la semantica in un modo che una macchina possa comprenderla.\\
Nel corso degli anni, nella letteratura sono state presentate diverse soluzioni: dagli albori di questo ambito, in cui i dati erano rappresentati in maniera strutturata e formale dalle ontologie, si è giunti alla rappresentazione superficiale ma efficiente del paradigma dei Linked (Open) Data. Questa sezione vuole introdurre agli standard di rappresentazione e recupero dei dati citati in questo lavoro.

\subsection[Resource Description Framework]{Resource Description Framework}
Il Resource Description Framework \cite{RDFspecification} (RDF d'ora in poi) è una serie di specifiche create da W3C (World Wide Web Consortium) che includono un modello per descrivere risorse web attraverso delle annotazioni, sotto forma di \textbf{triple}. Esse consistono in:
\[ < \text{soggetto},\ \text{predicato},\ \text{oggetto} > \]
Un insieme di queste triple, chiamato \textbf{grafo RDF}, può essere rappresentato come un grafo diretto etichettato in cui ogni tripla descrive un arco dal nodo soggetto al nodo oggetto.
\begin{figure}[h]
    \begin{minipage}{0.3\linewidth}
        \centering
        \begin{tikzpicture}[
                node distance = 15mm and 15mm,
                V/.style = {rounded corners, draw, fill=gray!30},
                every edge quotes/.style = {auto, font=\footnotesize, sloped}
            ]
            \begin{scope}[nodes=V]
                \node (1)   {Pizza};
                \node (2) [right=of 1]    {Margherita};
                \node (3) [below =of 2]    {Mozzarella};
                \node (4) [left=of 3]    {Vegetarian};
            \end{scope}
            \draw[->, ultra thick]   (2)  edge["isA"] (1)
            (2)  edge["madeOf"] (3)
            (4)  edge["canEat"] (2);
        \end{tikzpicture}
    \end{minipage}
    \hspace{5mm}
    \begin{minipage}{0.7\linewidth}
        \begin{alignat*}{4}
            G_1 = \{ (\  & \text{Margherita},\  &  & isA,      &  & \text{Pizza}        &  & ),  \\
            (\           & \text{Margherita},\  &  & madeOf,\  &  & \text{Mozzarella}\  &  & ),  \\
            (\           & \text{Vegetarian},\  &  & canEat,\  &  & \text{Mozzarella}\  &  & )\}
        \end{alignat*}
    \end{minipage}
    \caption{Un grafo RDF $G_1$ come grafo diretto (sx.) e insieme di triple (dx.)}
    \label{fig:grafoRDF}
\end{figure}

\noindent
Per essere più precisi, \textit{in RDF sia i nodi che i predicati sono risorse}, cioè delle entità nell'universo del discorso d'interesse. A seconda che la risorsa sia una stringa (come Margherita nella \autoref{fig:grafoRDF}) oppure un concetto più astratto è possibile identificarla in due modi diversi: tramite IRI oppure come un letterale. In ogni caso, definire una tripla RDF significa dire che la relazione indicata dal predicato vale fra le risorse indicate dal soggetto e dall'oggetto. Parliamo meglio di questi metodi d'identificazione:
\begin{description}
	\item[IRI (International Resource Identifier)]  Un formato generalizzato di URI che ricorre a un range più ampio di caratteri Unicode, che permette di identificare univocamente una risorsa. È importante sapere che solamente gli IRI vengono utilizzati per identificare i predicati in una tripla, e denotano una \textbf{proprietà}, cioè una risorsa che può essere vista come una relazione binaria.
	Un insieme di IRI destinati all'uso in un grafo RDF è chiamato \textbf{vocabolario RDF} e di solito tutti gli identificatori di uno stesso vocabolario condividono una sotto stringa iniziale comune, che prende il nome di \textbf{namespace}. Per migliorare la leggibilità dei documenti RDF, si utilizza un \textbf{prefisso}, che sostituisce il namespace per abbreviare la lunghezza del identificatore. Ad esempio, per l'IRI \verb|http://example.org/#Margherita| si potrebbe definire il namespace\\ \verb|http://example.org/#| con prefisso \verb|example|; in questo modo la risorsa Margherita può essere identificata con il più corto e leggibile \verb|example:Margherita|. Alcuni esempi concreti possono esseri trovati in \cite{RDFSspecification}.
	\item[Letterali] Rappresentano risorse nel senso di valori numerici, date o stringhe. La codifica di un qualsiasi valore di un letterale è come stringa Unicode, ma è possibile specificare il tipo, identificato da un IRI (e.g. il tipo delle date è \verb|xsd:date|), per permettere di risalire alla vera semantica del valore.
\end{description}
Esiste anche un altro termine che permette di affermare la presenza di un nodo per cui sussiste la relazione, senza nominarlo in maniera esplicita: il \textbf{blank node}. Per fare un paragone, possono essere considerati come variabili esistenziale, in cui il valore non è conosciuto ma si assume che sia presente.

\begin{definition}[Tripla RDF]
	Siano \textbf{I}, \textbf{L} e \textbf{B} rispettivamente gli insiemi infiniti e disgiunti uno a uno delle stringhe IRI, dei letterali e dei blank nodes. Una tripla RDF è una tupla $(s, p, o) \in \textbf{U}\ \cup \textbf{B} \times \textbf{U} \times \textbf{U}\ \cup \textbf{B}\ \cup \textbf{L}.$ in cui s è chiamato soggetto, p il predicato e o l'oggetto.
\end{definition}
\begin{definition}[Grafo RDF]
	Un grafo RDF è un insieme finito di triple RDF.
\end{definition}
\begin{definition}[Grafo RDF]
	Un grafo RDF è un insieme finito di triple RDF.
\end{definition}
IRI (International Resource Identifier),



\subsection{Ontologie}
\subsubsection[Ontologie come tipi per RDF]{Ontologie come tipi per annotazioni RDF}