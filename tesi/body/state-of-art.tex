\chapter[Stato dell'arte]{Stato dell'arte}
Come accennato nell'introduzione, l'obiettivo di questa tesi è di assistere la creatività umana tramite strumenti di cui è possibile dimostrare formalmente la bontà, in particolare attraverso i sistemi di tipo. In questo capitolo daremo una panoramica preliminare sulle motivazioni che ci hanno spinto a proporre questo nuovo filone di ricerca. Sposteremo poi l'attenzione sul vero e proprio protagonista del capitolo, il concetto di "tipo" nella letteratura del Web Semantico. Infine, daremo una recensione di come vengono utilizzati i tipi dei linguaggi funzionali (principalmente Haskell \cite{o2008real}, essendo un linguaggio funzionale \textit{puro}) per sviluppare tools e processare dati in questo campo di ricerca.

\section*{Motivazioni}
Il campo del Web Semantico è di natura un campo d'interesse interdisciplinare: oltre ai ricercatori delle discipline informatiche, il cui coinvolgimento principale è ottimizzare la deduzione della semantica, esistono anche contesti in cui si vorrebbe utilizzare questo nuovo paradigma di Web come un semplice strumento. Per esempio, quando il ragionamento formale sui linguaggi ontologici è risultato inefficiente per essere applicata a grandi mole di dati, si è adottato un paradigma più superficiale, abbandonando la semantica formale delle ontologie e preferendo i \textit{knowledge graphs}. Seguendo i trend degli utilizzatori, quindi, si è formata un'ampia collezione di letteratura e tecniche provenienti da svariati settori d'applicazione. Negli ultimi due decenni, il Web Semantico si è arricchito di meta-contenuti grazie agli sforzi di ricercatori e aziende. Contributi fondamentali dai campi del \textit{Natural Language Processing} e del \emph{Machine Learning} hanno permesso di sviluppare metodi per semi-automatizzare compiti complessi, come la costruzione di ontologie e knowledge graphs partendo da testi in linguaggio naturali o di aggiungere relazioni mancanti tra dati.\\
L'obiettivo finale di un Web accessibile agli agenti intelligenti è sempre più vicino, ma rimangono ancora dei problemi aperti. Uno tra questi è sicuramente la mancanza di una base formale su cui appoggiarsi. I knowledge graph  \cite{ehrlinger2016towards}. Un'ottima e recente recensione sommaria del Web Semantico può essere trovata in \cite{hitzler2021review}.

\section{Haskell}
I linguaggi funzionali possono essere molto utili negli ambiti in cui è richiesta l'elaborazione di una grande mole di dati, grazie alle loro funzionalità principali quali il pattern matching, la valutazione "non-strict"