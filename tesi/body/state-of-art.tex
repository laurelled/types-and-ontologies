\chapter[Stato dell'arte]{Stato dell'arte}
Come accennato nell'introduzione, l'obiettivo di questa tesi è di assistere la creatività umana tramite strumenti di cui è possibile dimostrare formalmente la bontà, in particolare attraverso i sistemi di tipo. In questo capitolo daremo una panoramica preliminare sulle motivazioni che ci hanno spinto ad indagare in questa direzione. Sposteremo poi l'attenzione sul vero e proprio protagonista del capitolo, il concetto di "tipo" nella letteratura del Web Semantico. Infine, daremo una recensione di come vengono utilizzati i tipi dei linguaggi funzionali (principalmente Haskell \cite{o2008real}, essendo un linguaggio funzionale \textit{puro}) per sviluppare tools e processare dati in questo campo di ricerca.
\\
\section[I problemi rilevati]{I problemi rilevati}
Il Web Semantico è di natura un campo d'interesse interdisciplinare: oltre ai poli di ricerca delle discipline informatiche, il cui coinvolgimento può essere riassunto nel miglioramento delle tecnologie e principi che lo costituiscono, esistono contesti in cui si vorrebbe utilizzare questo nuovo paradigma Web come un semplice strumento. Per fare un esempio, quando verso il 2006 le ontologie nello stato in cui erano si sono dimostrate difficili da mantere e riutilizzare, si è sviluppato un paradigma più superficiale chiamato \textit{Linked Open Data} \cite{hitzler2021review}. Essenzialmente l'organizzazione della conoscenza consisteva di grafi RDF che potevano essere collegati tra di loro qualora fossero presenti identificatori IRI (che identificano un nodo) in comune. In questo modo la collezione di tutti i grafi RDF collegati poteva essere inteso come un singolo grande grafo RDF, permettendo una maggiore integrazione e gestione dei dati rispetto alla gestione di un'ontologia.\\
Negli ultimi due decenni, il Web Semantico si è arricchito di tecnologie e strumenti grazie agli sforzi di ricercatori e aziende: contributi fondamentali dai campi del Natural Language Processing e del Machine Learning hanno permesso di sviluppare metodi per semi-automatizzare compiti complessi, come la costruzione di ontologie partendo da testi in linguaggio naturale o di aggiungere relazioni mancanti tra dati; nel 2012 Google ha lanciato il suo "knowledge graph", che si è diffuso rapidamente come tecnica di modellazione della conoscenza nel Web Semantico. Seguendo le necessità degli utilizzatori, quindi, si è formata un'ampia collezione di letteratura e tecniche per ogni settore d'applicazione. Per i lettori interessati alle influenze e stato attuale del Web Semantico, un'ottima e recente recensione può essere trovata qui \cite{hitzler2021review}.\\
L'obiettivo finale di un Web accessibile agli agenti intelligenti è sempre più vicino, ma rimangono ancora dei problemi aperti.
\begin{description}
	\item[Mancanza di definizioni formali.] Ad esempio, anche se è stato presentato per la prima volta 11 anni fa e sia largamente utilizzato, non è ancora stata data una definizione formale di cosa sia un knowledge graph, anche se sforzi verso questo risultato sono stati proposti \cite{ehrlinger2016towards}. La ragione principale che dovrebbe spingere la comunità del Web Semantico nel trovare delle definizione comuni è la disambiguazione dei termini, che funge da deterrente per la diffusione negli ambiti accademici e aziendali dei concetti e tecnologie del campo di ricerca. La poca consolidazione del vocabolario ostacola l'apprendimento delle persone interessate a svolgere ricerca nell'ambito del Web Semantico, rendendolo meno attraente.
\end{description}


\section[Sistemi di tipo per il Web Semantico]{Sistemi di tipo per il Web Semantico}
Pochi sistemi di tipo sono stati sviluppati per sviluppare linguaggi di programmazione.
RDFS e OWL permettono di categorizzare i dati, ma questo non è lontanamente vicino a un sistema di tipo.
Per quanto RDF permetta di descrivere risorse utilizzando un vocabolario arbitrariamente scelto dall'utente, non consente di definire la semantica del dominio d'interesse. Per questo scopo viene utilizzato RDF Schema (RDFS), un'estensione del vocabolario base di RDF costrutti per specificare proprietà e classi di risorse RDF. Permette cioè di specificare classi e proprietà, definirne una gerarchia e di assegnare delle risorse a una classe tramite la meta-relazione \verb|rdf:type|.