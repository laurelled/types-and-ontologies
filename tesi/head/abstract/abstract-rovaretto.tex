\chapter*{Abstract}

Lo scopo di questo lavoro è di esplorare possibili impieghi dei linguaggi funzionali con tipi per offrire strumenti formali e software che possano facilitare la rappresentazione e la manipolazione dei dati semanticamente annotati. Dopo un esame preliminare dello stato dell’arte su questi argomenti, il nostro lavoro si è concentrato sull’implementazione di un lambda calcolo con tipi statici presentato nella tesi di dottorato di Martin Leinberger dell’Università di Koblenz-Landau, proposto come linguaggio per la generazione di query su dati in formato Resource Description Framework (RDF) (che è lo standard W3C per la codifica, lo scambio e il riutilizzo di metadati strutturati) e relativa manipolazione dei risultati ottenuti. Infine, abbiamo speculato su possibili direzioni future di applicazioni dei linguaggi funzionali tipati alla rappresentazione della conoscenza.