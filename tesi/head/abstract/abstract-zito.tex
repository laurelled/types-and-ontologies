\chapter{Abstract}
I linguaggi funzionali tipati, grazie a concetti come le funzioni di ordine superiore, i data type e il pattern matching, permettono di programmare in modo naturale applicazioni modulari e di facile debugging.  Questa loro flessibilità e le garanzie di correttezza che offrono (grazie al controllo statico dei tipi) pare adatta per la costruzione di formalismi e strumenti per supportare un reasoning efficiente nell'ambito della Knowledge representation. Si pone sempre più, infatti, il problema di avere efficienza e, nel contempo, offrire semantiche espressive, per i dati in formato Resource Description Framework (RDF). Il formato RDF è lo strumento base proposto da W3C per la codifica, lo scambio e il riutilizzo di metadati strutturati (tramite grafi detti Knowledge graph) e consente l'interoperabilità semantica tra applicazioni che condividono le informazioni sul Web. Questo lavoro è uno studio esplorativo sullo stato dell'arte e sui possibili sviluppi futuri dell'applicazione dei linguaggi funzionali tipati nel contesto della Knowledge representation, in generale, e dei dati in formato RDF, in particolare.