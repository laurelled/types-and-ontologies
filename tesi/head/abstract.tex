\chapter{Abstract}
This work is a preliminary study of possible proposals in the direction of providing formal tools based on statically typed functional languages for the Semantic Web. The use of logical tools such as type systems for reasoning may seem to be a direction of research that goes against the current preference for sublogical methods, especially in machine learning, which are much more efficient. Although efficiency is a key aspect of a computational process, it is not the only factor.
This thesis offers:
\begin{enumerate}[I.]
	\item a significant, albeit incomplete, overview of the state of the art in research on the use of functional languages and static type systems in the context of the Semantic Web and knowledge representation. We also present some topics for which, to the best of our knowledge, there are still no models and tools using the functional paradigm and types, but which could benefit from their use. Lorenzo Bafunno was mainly responsible for this part.
	\item  an experiment to study and implement a proposed functional language with static types within the Semantic Web, namely Martin Leinberger's $\lambda_{DL}$ system introduced in his PhD thesis "Type-safe Programming for the Semantic Web" \cite{leinbergerphdthesis}. This experiment was mainly carried out by Emanuele Rovaretto and Andrea Zito.
	\item proposals for possible future directions in the use of statically typed functional languages for programming applications that manipulate ontologies, and the study of static type systems that guarantee interesting properties to programs that perform ontology-based computations, certified by the type system itself. This part is the result of a choral effort supported by some valuable conversations with Marco Antonio Stranisci, Rossana Damiano and Antonio Lieto of the Department of Computer Science of the University of Turin.
\end{enumerate}
\noindent
Regarding future directions, it should be noted that currently the vast majority of ontology-based reasoning is performed at run-time. Java libraries such as the OWL API \cite{OWLAPI} are used to represent ontology axioms in object-oriented programming languages and to exploit them for reasoning. If run-time errors occur due to the ontology, there is no formal guarantee that the programme can finish with the desired result. It is therefore proposed to use static type systems, where the use of types that are checked at compile time could be an answer to efficiency problems for certain properties that would make sense to check a priori and/or in the case of large amounts of data. Martin Leinberger's PhD thesis \cite{leinbergerphdthesis} goes in this direction by proposing a $\lambda$-calculus with static types to decide at compile time whether a SPARQL query is inhabited, i.e. whether it will produce a result when interpreted, to ensure its use at run-time.